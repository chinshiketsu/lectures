%-*- coding: UTF-8 -*-
% grad_div_curl.tex
% 梯度散度和旋度
\documentclass[UTF8]{ctexart}  
\usepackage{graphicx}
\usepackage{amsmath}

% 在maketitle部分输出出来
\title{梯度、散度和旋度小结}
\author{陈斯杰}
\date{\today}

\bibliographystyle{plain}  %参考文献格式
\newtheorem{deff}{定义}
\newtheorem{thm}{定理}


% 前面都是导言区,后面是正文区,正文区才是直接输出的部分
\begin{document}
	\maketitle
	\begin{abstract}
	\zihao{-4} \kaishu
	这是一篇关于梯度、散度和旋度的一篇小总结
	\end{abstract}
	
	
	\tableofcontents
	\section{梯度}
		\subsection{二元函数情形}
		\begin{deff}
			二元函数的梯度是这样一个向量:其方向是这点的方向导数取最大值的方向,其模就是方向导数的最大值
			\begin{align*}
			\textbf{grad}\ f(x_0,y_0)&=\nabla f(x_0,y_0)\\
									 &=f_x(x_0,y_0) \textbf{i} + f_y(x_0,y_0) \textbf{j}\\
									 &=(f_x(x_0,y_0), f_y(x_0,y_0))\\
			\end{align*}
		
		\end{deff}
	
		\subsection{三元函数情形}
			三元函数与二元函数类似,$\nabla f$是这样一个向量:其方向是这点的方向导数取最大值的方向,其模就是方向导数的最大值
			\begin{align*}
			\textbf{grad}\ f(x_0,y_0,z_0)&=\nabla f(x_0,y_0,z_0)\\
									 &=f_x(x_0,y_0,z_0) \textbf{i} + f_y(x_0,y_0,z_0) \textbf{j} + f_z(x_0,y_0,z_0) \textbf{k}\\
									 &=(f_x(x_0,y_0), f_y(x_0,y_0), , f_z(x_0,y_0))\\
			\end{align*}
			对于函数等值面$f(x,y,z)=c$上的一点$(x_0,y_0,z_0)$, \\
			梯度\ $\nabla f(x_0,y_0,z_0)$的方向就是等值面在这一点的法线的方向n,\\
			梯度的模\ $||\nabla f(x_0,y_0,z_0)||$就是函数沿着法线方向的方向导数
			$\frac{\partial f}{\partial n}$
		
		\subsection{Nabla算子}
			$\nabla= \frac{\partial }{\partial x} \textbf{i}
					+\frac{\partial }{\partial y} \textbf{j}			
					+\frac{\partial }{\partial z} \textbf{k}
			$
			称为三维的向量微分算子,或者Nabla算子\\			
 			 
			$\nabla f= \frac{\partial f}{\partial x} \textbf{i}
					+\frac{\partial f}{\partial y} \textbf{j}			
					+\frac{\partial f}{\partial z} \textbf{k}
			$
		\subsection{梯度的场论意义}
			\begin{deff}
			建立一个对应关系:对于空间区域G内每个点M,都有一个确定的\textbf{标量}$f(M)$与之对应,则称在G中建立了一个数量场。(如温度场、密度场)
			\end{deff}
			
			\begin{deff}
			建立一个对应关系:对于空间区域G内每个点M,都有一个确定的\textbf{向量}$F(M)$与之对应,则称在G中建立了一个向量场。(如速度场、力场)
			  
			可表示为$F(M)=P(M) \textbf{i}+Q(M) \textbf{j}+R(M) \textbf{k}$
			\end{deff}
  
  
			若$F(M)= \nabla f(M)$,则称$F(M)$为势场,$f(M)$为$F(M)$的势函数。\\
			例如引力势$\frac{m}{r}$的梯度$\textbf{grad}\frac{m}{r}$称为引力场\\
			一孤立点电荷Q,在距Q为r的A点电势为$\phi=k \frac{Q}{r}$\\
			$\frac{\partial{\phi}}{\partial{r}}= $
			
	\bibliography{math} %提示从文献数据库math中获得文献信息,打印参考文献表
\end{document}