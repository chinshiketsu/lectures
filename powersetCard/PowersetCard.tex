\documentclass[UTF8,10pt,a4paper]{ctexart}
\usepackage[utf8]{inputenc}
\usepackage{amsmath}
\usepackage{amsfonts}
\usepackage{amssymb}
\usepackage{graphicx}

\newtheorem{deff}{定义}

\begin{document}
\section{集合与其幂集不等势}
	\subsection{幂集}
		一个集合的所有子集构成的集族称为幂集。
		 		 
		集合$A$的幂集记作$P(A)$或者$2^A$
	\subsection{等势}
		\begin{deff}
			双射(Bijection),又称一一映射。
              			
			对于一个映射$f:A \mapsto B$,当且仅当:
			 
			对于$\forall b\in B$,存在唯一的$a\in A$满足$f(a)=b$ 时,
			
			我们称其为双射。
		\end{deff}
	
		\begin{deff}
			在集合$A$和集合$B$之间,如果能构建一个双射$f:A \mapsto B$,则称$A$和$B$等势。
		\end{deff}

	\subsection{证明:任意集合与该集合的幂集不等势}
		\subsubsection{正式证明}
			假设集合$A$与其幂集$P(A)$等势,那么存在一个双射$f:A \mapsto P(A)$。\\
			设$B=\{a| a\in A\ \ and\ \ a\notin f(a) \}$,由于$B$中所有元素$a$都满足$a\in A$,所以$B$是$A$的子集。
			 
			根据幂集的定义:“幂集是一个集合所有子集的集族”, 所以 $B$是$P(A)$的一个元素,即$B \in P(A)$。
			
			由我们的假设,双射$f$使得存在唯一的一个元素$\ b\in A\ $满足$\ f(b)=B$。
			
			那么现在来看看$b$是否属于集合$B$:
			
			假如$b\in B$,$b$一定必须要集合$B$的定义:$b \notin f(b)$,而$f(b)=B$,从而推出$b \notin B$, 矛盾,所以不成立。
			
			假如$b\notin B$,把前面的推论$B=f(b)$代入,得:$b \notin f(b)$。然而这又使得$b$满足了$B$的定义
			“$b\in A\ \ and\ b \notin f(b)$”,从而推出$b \in B$,矛盾,所以不成立。
			
			$b\in B$和$b\notin B$是两个互相矛盾的命题,不能同时为假,否则违反排中律。因此原假设“集合$A$与其幂集$P(A)$等势”不成立。
			
			证毕
		\subsubsection{讲故事证明}
			在数学王国Byteland里面,三拍干部领导班子出了一个点子:在生产队$A$和超生产队$P(A)$之间开展手拉手结对子活动。
			超生产队
\end{document}